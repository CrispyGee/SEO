\chapter{Sitemap}
\subchapter{Definition}
Eine Sitemap ist eine Liste aller URLs einer Webseite. Es ist also eine Übersicht aller verfügbarer Inhalte der Webseite und fungiert ebenfalls als strukturelle Grundlage, um den Aufbau der Webseite besser zu verstehen.\\
Man kann Suchmaschinen eine Sitemap der Webseite zur Verfügung stellen. 
Eine HTML(Hypertext Markup Language)-Sitemap sind speziell für den User gedacht und ist ein grafisch übersichtliches Inhaltsverzeichnis der Webseite. Sie trägt zur Usability der Webseite bei.
Eine XML(Extensible Markup Language)-Sitemap ist ein strukturiertes Inhaltsverzeichnis und trägt nichts zur Usability bei. Sie ist jedoch speziell für Suchmaschinen von hoher Bedeutung.\\
Somit können durch eine XML-Sitemap zum Beispiel Unterseiten gefunden werden, welche ansonsten unter Umständen eventuell übersehen werden. Zum Beispiel, wenn eine URL neu ist und auf der Seite nur durch wenige Links darauf verwiesen wird.  Oder wenn eine URL sehr tief innerhalb der Seitenstruktur verschachtelt ist. 
Sie bietet zwar keine Garantie dafür, dass alle Seiten gecrawlt werden, aber hilft dafür beim Finden neuer und/oder verschachtelter Seiten.
\subchapter{Vorgehen}
Die Sitemap manuell zu erstellen wäre mit einem großen Aufwand verbunden und erfordert viel Sorgfalt, damit diese immer aktuell bleibt, daher entschieden wir uns ein Tool zum Erstellen zu Nutzen. Wir nutzten hierbei dsa Tool \glqq QLue Sitemap”, welche eine Sitemap sowohl in HTM-, als auch in XML-Format erstellt.
Wie gewohnt kann die Erweiterung über das Joomla! Backend installiert werden. Anschließend müssen drei Plugins aktiviert werden: QMap - Content, QMap - Categories und QMap - Menu, woraufhin der Eintrag \glqq Qlue Sitemap” im Komponenten-Menü erscheint. Dort kann man nun Sitemap erzeugen. Die Erweiterung sammelt automatisch alle freigegebenen Seiten der Joomla! Installation und erstellt die Sitemap.
Für den Nutzer kann man diese Sitemap nun als neuen Menüpunkt auf der Seite integrieren worunter man die HTML-Version vorfinden wird. Deutlich interessanter für uns war jedoch die XML-Version,welche man über die URL aufrufen kann:
http://lolx.de /sitemap?format=xml
Nach Erstellung der Sitemap müssen wir Google noch aufmerksam machen, dass diese existiert. Dies ist dank des Zugriff auf Google Webmaster-Tools sehr simpel. 
Unter Crawling findet sich der Menüpunkt \glqq Sitemaps”, welcher eine leere Seite enthielt.
Ein Klick auf Sitemap Hinzufügen/Testen und die Angabe des oben aufgeführten Links fügte die Sitemap hinzu.
\\ \\ BILD \\ \\
Nach Abschluss dieser Aktion finden sich Daten vor, wie sie auf Abbildung XYZ zu sehen sind. Google Webmaster-Tools zeigt uns alle Sitemaps an - in unserem Fall nur eine - und stellt sofort einige Informationen dazu zur Verfügung. Interessant ist hierbei vor allem die Anzahl der eingereichten Verweise bzw. Seiten und der Status Indexiert. Leider braucht auch hier SEO wieder seine Zeit und die Sitemap ist Google nun zwar bekannt gemacht worden, der Google Crawler muss diese aber zuerst noch indexieren. Eine Zeitangabe zum Abschluss des Crawling Vorgangs kann hierbei nicht getroffen werden.
Mit dem Hinzufügen in Google Webmaster-Tools ist ein erster Schritt getan. Im weiteren Verlauf sollte die Sitemap jedoch auch in der robots.txt bekannt gemacht werden. Diese findet man im Stammverzeichnis der Joomla! Installation. An den Anfang dieser Datei setzt man nun einen Verweis auf die XML-Sitemap:
sitemap: http://lolx.de/index.php/sitemap?format=xml
Der Crawler weiß nun bei jedem Besuch wo er die Sitemap finden kann und wird diese  durchsuchen. Zur Sicherheit sollte man außerdem seine robots.txt testen. Dazu bietet Google ebenfalls eine Funktion im Menübereich Crawling. Die Option robots.txt-Tester überprüft die Datei auf Fehler. Man sollte die Datei außerdem nach Bearbeitung nochmals über den Senden-Button an Google schicken, einige MInuten warten, die Seite neu laden und anschließend manuell überprüfen ob die korrekte robots.txt angezeigt wird.
\\ \\ BILD \\ \\
Wie in Abbildung XYZ zu sehen ist, findet sich der Verweis zur Sitemap nun in der robots.txt und diese enthält weder Fehler noch Warnungen. Damit sollte die Sitemap nun erfolgreich zur Optimierung beitragen und wir können einen Schritt weitergehen.


