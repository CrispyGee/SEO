\chapter{Einleitung}


Will ein gewöhnlicher Computer-Nutzer heute etwas im Internet nachschlagen, so nutzt dieser kaum noch URLs. 
Er gibt einfach seinen gewünschten Begriff oder seine Frage in eine Suchmaschinen ein und erhält sofort eine umfassende Liste mit möglichen Ergebnissen dazu. Via Klick wählt er nun das Ergebnis aus, welches auf seine Suchanfrage passt und gelangt somit zu seiner gewünschten Homepage. Ist dies nicht der Fall, so geht er zurück und wählt ein weiteres Ergebnis aus der Liste aus oder startet eine neue, genauere Suchanfrage, um zu seiner gewünschten Seite zu gelangen.
Hat man nun eine eigene Homepage und möchte, dass Nutzer auf diese aufmerksam werden und sie nutzen, ist es heutzutage ein wichtiger Bestandteil, dass diese bei Suchmaschinen unter den richtigen Suchbegriffen in der Ergebnisliste sehr weit oben gelistet ist.
Die Suchmaschinen arbeiten bei der Ergebnisfindung mit Algorithmen, welche nicht öffentlich zugänglich sind. Dies erschwert die Webseitenoptimierung, da man nicht weiß, wie genau man seine Webseite überarbeiten muss, um diese hoch listen zu lassen. 
Einige der Faktoren, nach welchen sie arbeiten sind jedoch im Laufe der Zeit erkannt worden, bei anderen wird noch viel spekuliert. Um diese zu erkennen, muss analysiert werden, wie der Traffic auf der Webseite ist, woher die Nutzer kommen und wie sich dies verändert, nachdem die Webseite überarbeitet worden ist.
Ein wichtiger Faktor bei Suchmaschinen ist ebenfalls, dass dessen Algorithmen andauernd optimiert und angepasst werden. Somit muss der Webseitenersteller seine Webseite ebenfalls andauernd an die Algorithmen anpassen, um weiterhin in der Ergebnisliste vorne zu bleiben. Suchmaschinenoptimierung ist also ein laufender Prozess.
Wir entschieden uns dafür, für das Wahlpflichtfach “Search Engine Optimization” eine Homepage für Suchmaschinen zu optimieren und uns mit der Thematik der Suchmaschinen auseinanderzusetzen.
Zunächst mussten wir uns auf eine Homepage, welche wir analysieren und verbessern wollten, einigen.
Die Homepage “www.lolx.de” wurde vor einiger Zeit von einem Teammitglied erstellt. Sie wurde bisher nicht mithilfe von Search-Engine-Optimization verbessert, hatte aber schon einen thematischen Inhalt, der potentiell von einer größere Nutzerbasis verwendet werden könnte, weshalb wir uns für sie entschieden.
Es handelt sich um eine Fanseite für ein bekanntes Online-Spiel und enthält unter anderem ein Forum, in welchem sich die Spieler austauschen können.
Obwohl die Seite noch nicht optimiert wurde, war sie bereits bei den drei Suchbegriffen “Flyff Forum Fanseite” auf Seite 2 der Ergebnisliste bei Google.
Unser Ziel war es, die Seite  mit Hilfe von diversen Search-Engine-Optimization-Techniken bei diesen drei Suchbegriffen in der Ergebnisliste nach vorne zu bringen (möglichst auf Platz 1).
