\chapter{Grundlagen}

Um die Webseite zu verbessern, mussten wir zunächst einmal feststellen, wie der derzeitige Stand war. Wir integrierten daher Google Analtics und Google Webmaster Tools, mit welchen wir feststellten, woher die Nutzer kamen, wie lange sie auf der Seite waren und was wir noch für die Optimierung verändern konnten. Auf diese Tools und unsere Verwendung davon wird nun genauer eingegangen.

Google Analytics

Google Analytics ist ein Dienst, welcher den Datenverkehr von Webseiten analysiert und dem Besitzer Auskunft über die Besucher seine Seite gibt. Der Dienst wurde von der Urchin Software Corporation entwickelt und 2005 von Google Inc. übernommen und weiterentwickelt.
Sie misst den gesamten Traffic der Seite und gibt dem Nutzer detaillierte Auskunft darüber.
Mithilfe von Google Analytics kann man unter anderem die Herkunft der Besucher, wie lange sie auf der Seite geblieben sind, und weitere wichtige Informationen erkennen.
Das Integrieren von Google Analytics ging sehr einfach und schnell. Da die Webseite mit Joomla aufgebaut ist, verwendeten wir das Plug-In “Advanced Google Analytics for Joomla” von “DEConf.com”, um Google Analtics zu integrieren und aktivierten dieses. 
Das Forum der Seite ist nicht mit Joomla aufgebaut, sondern mit phpBB3, weshalb wir dort ebenfalls Google Analytics integrierten. Hierfür mussten wir einen Code-Ausschnitt, welcher auf Google Analytics zur Verfügung gestellt wird, in die Seite integrieren. Dies ging ebenfalls sehr schnell und einfach.
Anschließend liessen sich bereits erste Ergebnisse erkennen und man sah erste Auswertungen.


Google Webmaster Tools

Google Webmaster Tools ist ein Webservice von Google Inc, über welchen man Informationen über die eigene Webseite erhält, welche nicht öffentlich verfügbar sind. 
Sie beschränkt sich dabei jedoch vollständig auf die Google-Websuche und ignoriert dabei sämtliche anderen Suchmaschinen. Für die Suchmaschine bing.com von Microsoft werden jedoch ebenfalls Webmaster-Tools angeboten.
Mithilfe von Google Webmaster Tools kann man zum Beispiel sehen, welche Webseite auf die eigene Seite verlinken, ob und welche Probleme beim Crawling festgestellt wurden, und zu welchen Suchbegriffen in der Google-Suche die eigene Seite als Ergebnis gefunden wird.

Um Google Webmaster Toos zu verwenden, mussten wir uns als Webmaster der Webseite identifizieren. Da wir jedoch zuvor bereits Google Analytics integriert hatten, konnten wir die Identifikation für Google Webmaster Tools übergehen, da wir bei Google Analytics bereits als Webmaster identifiziert wurden.

