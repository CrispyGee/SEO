\chapter{Ausblick}

Ausblick
Nach unserer Verbesserung können wir nun die Veränderungen der Webseite mithilfe von Google Analytics analysieren und auswerten.

Zielgruppe
Die durchschnittiche Sitzungsdauer beträgt ca. 4 Minuten und enthält eine sehr hohe Absprungrate von knapp 47%, was dem allgemeinen Verhalten in einem Forum entspricht (Ein Nutzer googelt eine Frage, kommt auf das Forum, liest kurz die Beiträge und verlässt nach gefundener Antwort das Forum wieder).


Da das Forum sehr viele Stammnutzer (unter anderem Mitglieder diverser Gilden) hat, war es zu Beginn fast ausschließlich von widerkehrenden Nutzern besucht. Nach der Optimierung ist dies sehr ausgeglichen.


Das durchschnittliche Alter trifft sich mit der Zielgruppe sehr gut, da Online-Rollenspiele meist von jüngeren Menschen gespielt werden.



Auch das Geschlecht der Nutzer ist sehr ausgeglichen, es sind knapp mehr männliche Nutzer als weibliche.




Überraschend war das Ergebnis von der Auswertung der Interessen der Zielgruppe. Dort hatten wir erwartet, Computer und Elektronikzubehör mit einem deutlich höheren Prozentsatz bemessen zu sein, stattdesssen liegt dieser nur bei knapp 4%.

Die Herkunft der Nutzer war sehr überraschend. Da es ein sich um ein deutschsprachiges Forum handelt, erwarteten wir ausschließlich Nutzer aus Deutschland. Während der Großteil aus Deutschland kommt, gibt es jedoch auch Sitzungen zum Beispiel aus den United States (81 Sitzungen).
Die meisten Sitzungen waren jedoch wie erwartet aus Deutschland.


Auffällig an den Nutzungen in Deutschland sind hierbei die zwei Ballungszentren Baden-Würtemberg und Nordrhein-Westfalen.

Die Hauptsprache der Nutzer ist daher auch Deutsch.

Bei der Browserwahl, sowie der Nutzung von mobilen Geräten ist ebenfalls keine unerwartete Überraschung zu erkennen.




Wichtig war der Verhaltensfluss. Hierbei erkennt man, dass maximal 3 Klicke notwendig sind, um das gewünschte Ziel zu erreichen. Nichtsdestotrotz erkennt man hier leider eine sehr hohe Absprungrate, befindet sich das gewünschte Ziel nicht auf der Seite, auf der man zuerst gelandet ist.





